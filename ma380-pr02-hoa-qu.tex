% Options for packages loaded elsewhere
\PassOptionsToPackage{unicode}{hyperref}
\PassOptionsToPackage{hyphens}{url}
%
\documentclass[
]{article}
\usepackage{amsmath,amssymb}
\usepackage{iftex}
\ifPDFTeX
  \usepackage[T1]{fontenc}
  \usepackage[utf8]{inputenc}
  \usepackage{textcomp} % provide euro and other symbols
\else % if luatex or xetex
  \usepackage{unicode-math} % this also loads fontspec
  \defaultfontfeatures{Scale=MatchLowercase}
  \defaultfontfeatures[\rmfamily]{Ligatures=TeX,Scale=1}
\fi
\usepackage{lmodern}
\ifPDFTeX\else
  % xetex/luatex font selection
\fi
% Use upquote if available, for straight quotes in verbatim environments
\IfFileExists{upquote.sty}{\usepackage{upquote}}{}
\IfFileExists{microtype.sty}{% use microtype if available
  \usepackage[]{microtype}
  \UseMicrotypeSet[protrusion]{basicmath} % disable protrusion for tt fonts
}{}
\makeatletter
\@ifundefined{KOMAClassName}{% if non-KOMA class
  \IfFileExists{parskip.sty}{%
    \usepackage{parskip}
  }{% else
    \setlength{\parindent}{0pt}
    \setlength{\parskip}{6pt plus 2pt minus 1pt}}
}{% if KOMA class
  \KOMAoptions{parskip=half}}
\makeatother
\usepackage{xcolor}
\usepackage[margin=1in]{geometry}
\usepackage{color}
\usepackage{fancyvrb}
\newcommand{\VerbBar}{|}
\newcommand{\VERB}{\Verb[commandchars=\\\{\}]}
\DefineVerbatimEnvironment{Highlighting}{Verbatim}{commandchars=\\\{\}}
% Add ',fontsize=\small' for more characters per line
\usepackage{framed}
\definecolor{shadecolor}{RGB}{248,248,248}
\newenvironment{Shaded}{\begin{snugshade}}{\end{snugshade}}
\newcommand{\AlertTok}[1]{\textcolor[rgb]{0.94,0.16,0.16}{#1}}
\newcommand{\AnnotationTok}[1]{\textcolor[rgb]{0.56,0.35,0.01}{\textbf{\textit{#1}}}}
\newcommand{\AttributeTok}[1]{\textcolor[rgb]{0.13,0.29,0.53}{#1}}
\newcommand{\BaseNTok}[1]{\textcolor[rgb]{0.00,0.00,0.81}{#1}}
\newcommand{\BuiltInTok}[1]{#1}
\newcommand{\CharTok}[1]{\textcolor[rgb]{0.31,0.60,0.02}{#1}}
\newcommand{\CommentTok}[1]{\textcolor[rgb]{0.56,0.35,0.01}{\textit{#1}}}
\newcommand{\CommentVarTok}[1]{\textcolor[rgb]{0.56,0.35,0.01}{\textbf{\textit{#1}}}}
\newcommand{\ConstantTok}[1]{\textcolor[rgb]{0.56,0.35,0.01}{#1}}
\newcommand{\ControlFlowTok}[1]{\textcolor[rgb]{0.13,0.29,0.53}{\textbf{#1}}}
\newcommand{\DataTypeTok}[1]{\textcolor[rgb]{0.13,0.29,0.53}{#1}}
\newcommand{\DecValTok}[1]{\textcolor[rgb]{0.00,0.00,0.81}{#1}}
\newcommand{\DocumentationTok}[1]{\textcolor[rgb]{0.56,0.35,0.01}{\textbf{\textit{#1}}}}
\newcommand{\ErrorTok}[1]{\textcolor[rgb]{0.64,0.00,0.00}{\textbf{#1}}}
\newcommand{\ExtensionTok}[1]{#1}
\newcommand{\FloatTok}[1]{\textcolor[rgb]{0.00,0.00,0.81}{#1}}
\newcommand{\FunctionTok}[1]{\textcolor[rgb]{0.13,0.29,0.53}{\textbf{#1}}}
\newcommand{\ImportTok}[1]{#1}
\newcommand{\InformationTok}[1]{\textcolor[rgb]{0.56,0.35,0.01}{\textbf{\textit{#1}}}}
\newcommand{\KeywordTok}[1]{\textcolor[rgb]{0.13,0.29,0.53}{\textbf{#1}}}
\newcommand{\NormalTok}[1]{#1}
\newcommand{\OperatorTok}[1]{\textcolor[rgb]{0.81,0.36,0.00}{\textbf{#1}}}
\newcommand{\OtherTok}[1]{\textcolor[rgb]{0.56,0.35,0.01}{#1}}
\newcommand{\PreprocessorTok}[1]{\textcolor[rgb]{0.56,0.35,0.01}{\textit{#1}}}
\newcommand{\RegionMarkerTok}[1]{#1}
\newcommand{\SpecialCharTok}[1]{\textcolor[rgb]{0.81,0.36,0.00}{\textbf{#1}}}
\newcommand{\SpecialStringTok}[1]{\textcolor[rgb]{0.31,0.60,0.02}{#1}}
\newcommand{\StringTok}[1]{\textcolor[rgb]{0.31,0.60,0.02}{#1}}
\newcommand{\VariableTok}[1]{\textcolor[rgb]{0.00,0.00,0.00}{#1}}
\newcommand{\VerbatimStringTok}[1]{\textcolor[rgb]{0.31,0.60,0.02}{#1}}
\newcommand{\WarningTok}[1]{\textcolor[rgb]{0.56,0.35,0.01}{\textbf{\textit{#1}}}}
\usepackage{longtable,booktabs,array}
\usepackage{calc} % for calculating minipage widths
% Correct order of tables after \paragraph or \subparagraph
\usepackage{etoolbox}
\makeatletter
\patchcmd\longtable{\par}{\if@noskipsec\mbox{}\fi\par}{}{}
\makeatother
% Allow footnotes in longtable head/foot
\IfFileExists{footnotehyper.sty}{\usepackage{footnotehyper}}{\usepackage{footnote}}
\makesavenoteenv{longtable}
\usepackage{graphicx}
\makeatletter
\def\maxwidth{\ifdim\Gin@nat@width>\linewidth\linewidth\else\Gin@nat@width\fi}
\def\maxheight{\ifdim\Gin@nat@height>\textheight\textheight\else\Gin@nat@height\fi}
\makeatother
% Scale images if necessary, so that they will not overflow the page
% margins by default, and it is still possible to overwrite the defaults
% using explicit options in \includegraphics[width, height, ...]{}
\setkeys{Gin}{width=\maxwidth,height=\maxheight,keepaspectratio}
% Set default figure placement to htbp
\makeatletter
\def\fps@figure{htbp}
\makeatother
\setlength{\emergencystretch}{3em} % prevent overfull lines
\providecommand{\tightlist}{%
  \setlength{\itemsep}{0pt}\setlength{\parskip}{0pt}}
\setcounter{secnumdepth}{-\maxdimen} % remove section numbering
\ifLuaTeX
  \usepackage{selnolig}  % disable illegal ligatures
\fi
\IfFileExists{bookmark.sty}{\usepackage{bookmark}}{\usepackage{hyperref}}
\IfFileExists{xurl.sty}{\usepackage{xurl}}{} % add URL line breaks if available
\urlstyle{same}
\hypersetup{
  pdftitle={MA 380 Project Two},
  pdfauthor={Quan Hoang},
  hidelinks,
  pdfcreator={LaTeX via pandoc}}

\title{MA 380 Project Two}
\usepackage{etoolbox}
\makeatletter
\providecommand{\subtitle}[1]{% add subtitle to \maketitle
  \apptocmd{\@title}{\par {\large #1 \par}}{}{}
}
\makeatother
\subtitle{Bike Sharing Analysis}
\author{Quan Hoang}
\date{\textbf{Due Date: Friday November 17, 2023, 11:59 PM EST}}

\begin{document}
\maketitle

\hypertarget{business-problem}{%
\section{Business Problem}\label{business-problem}}

Many cities and towns now provide locked bikes throughout their
neighborhoods. Customers sign-up for a sharing contract and they are
able to pick up a bike in one location, and ride it to a different
location to return it. You have been hired to help a town understand
when are customers using the bikes. The town administration would like
to create a model that will predict the number of bikes used in a given
hour and the locations that the bikes are moving from and to.

The town's Information Technology department has prepared a data set for
you. The data dictionary is provided below. This is the only data you
have available for your analysis.

\hypertarget{data-dictionary}{%
\subsection{Data Dictionary}\label{data-dictionary}}

\begin{longtable}[]{@{}
  >{\raggedright\arraybackslash}p{(\columnwidth - 2\tabcolsep) * \real{0.2133}}
  >{\raggedright\arraybackslash}p{(\columnwidth - 2\tabcolsep) * \real{0.7867}}@{}}
\toprule\noalign{}
\begin{minipage}[b]{\linewidth}\raggedright
Variable
\end{minipage} & \begin{minipage}[b]{\linewidth}\raggedright
Description
\end{minipage} \\
\midrule\noalign{}
\endhead
\bottomrule\noalign{}
\endlastfoot
season.code & Season (1 = Winter, 2 = Spring, 3 = Summer, 4 = Fall \\
year.code & Year indicator (0 = 2011, 1 = 2012) \\
hour & Hour (integer 0 to 23) \\
holiday.code & Indicator of holiday (0 = No, 1 = Yes) \\
weekday.code & Day of the week (0 = Sunday, 1 = Monday, \ldots, 6 =
Saturday) \\
weathersit.code & Weather situation (1 = Clear/Partly Cloudy, 2 = Mist,
3 = Rain or Snow) \\
temp & Normalized temperature in Celsius. {[}(t - t\_min)/(t\_max -
t\_min), t\_min = -9, t\_max = 39{]} \\
humidity & Normalized humidity. Values are divided by 100 (max
possible) \\
windspeed & Normalized wind speed. Values are divided by 67 (max
possible) \\
bikes & Count of rental bikes in each hour \\
\end{longtable}

\hypertarget{task-0-0-points}{%
\subsection{Task 0 (0 points)}\label{task-0-0-points}}

Read the data and provide appropriate types to the variables in the data
set. Many of the variables are code as integers making it difficult to
know what their values mean. Create new variables that are more human
friendly. Note that in later tasks you may need to modify the variable
types given here. Use function \texttt{read\_csv()} from the
\texttt{tidyverse} package and set the argument \texttt{col\_types}
appropriately.

\begin{center}\rule{0.5\linewidth}{0.5pt}\end{center}

\begin{Shaded}
\begin{Highlighting}[]
\CommentTok{\# import data}
\NormalTok{bike }\OtherTok{\textless{}{-}} \FunctionTok{read.csv}\NormalTok{(}\StringTok{"/Users/brenthoang/Library/CloudStorage/OneDrive{-}BentleyUniversity/MA 380/Project 2/ma380{-}pr02{-}bike{-}sharing.csv"}\NormalTok{)}

\CommentTok{\# quick glimpse}
\FunctionTok{glimpse}\NormalTok{(bike)}
\end{Highlighting}
\end{Shaded}

\begin{verbatim}
## Rows: 17,376
## Columns: 10
## $ season.code     <int> 1, 1, 1, 1, 1, 1, 1, 1, 1, 1, 1, 1, 1, 1, 1, 1, 1, 1, ~
## $ year.code       <int> 0, 0, 0, 0, 0, 0, 0, 0, 0, 0, 0, 0, 0, 0, 0, 0, 0, 0, ~
## $ hour            <int> 0, 1, 2, 3, 4, 5, 6, 7, 8, 9, 10, 11, 12, 13, 14, 15, ~
## $ holiday.code    <int> 0, 0, 0, 0, 0, 0, 0, 0, 0, 0, 0, 0, 0, 0, 0, 0, 0, 0, ~
## $ weekday.code    <int> 6, 6, 6, 6, 6, 6, 6, 6, 6, 6, 6, 6, 6, 6, 6, 6, 6, 6, ~
## $ weathersit.code <int> 1, 1, 1, 1, 1, 2, 1, 1, 1, 1, 1, 1, 1, 2, 2, 2, 2, 2, ~
## $ temp            <dbl> 0.24, 0.22, 0.22, 0.24, 0.24, 0.24, 0.22, 0.20, 0.24, ~
## $ humidity        <dbl> 0.81, 0.80, 0.80, 0.75, 0.75, 0.75, 0.80, 0.86, 0.75, ~
## $ windspeed       <dbl> 0.0000, 0.0000, 0.0000, 0.0000, 0.0000, 0.0896, 0.0000~
## $ bikes           <int> 16, 40, 32, 13, 1, 1, 2, 3, 8, 14, 36, 56, 84, 94, 106~
\end{verbatim}

\hypertarget{task-1-10-points}{%
\subsection{Task 1 (10 points)}\label{task-1-10-points}}

Assess whether or not the data you have will help you address the
business problem that the town is facing. In your assessment be sure to
clearly mention how the data you have been given will be useful or not
in addressing the two concerns that the town's administration has.

\begin{center}\rule{0.5\linewidth}{0.5pt}\end{center}

\begin{enumerate}
\def\labelenumi{\arabic{enumi}.}
\item
  The business problem is to create a model that will predict the number
  of bikes used in a given hour and the locations that the bikes are
  moving from and to. However, as we explore the provided dataset, there
  is no location identifier column(s) that helps to identify the
  locations of the bike. Therefore, we could not create a model to
  predict the number of bikes used based on the bike location.
\item
  The dataset does provide useful insight regarding bike usage in a
  specific time. Columns such as season.code, hour and holiday.code can
  be useful in the time aspect of bike usage. However, it also poses
  some correlation issue between some of the variables. We can deduct
  that there will be correlation between season.code, windspeed, temp
  and humidity. Taking these correlation into account will be important
  in the final model.
\end{enumerate}

\hypertarget{task-2-10-points}{%
\subsection{Task 2 (10 points)}\label{task-2-10-points}}

Which variables should be treated as categorical? Provide an explanation
for choosing these variables as categorical and change their types in
your data set.

\begin{center}\rule{0.5\linewidth}{0.5pt}\end{center}

\begin{Shaded}
\begin{Highlighting}[]
\CommentTok{\# characterize season and weekday}
\NormalTok{bike}\SpecialCharTok{$}\NormalTok{season }\OtherTok{\textless{}{-}} \FunctionTok{factor}\NormalTok{(}\FunctionTok{c}\NormalTok{(}\StringTok{"Winter"}\NormalTok{, }\StringTok{"Spring"}\NormalTok{, }\StringTok{"Summer"}\NormalTok{, }\StringTok{"Fall"}\NormalTok{)[bike}\SpecialCharTok{$}\NormalTok{season.code],}
                      \AttributeTok{levels =} \FunctionTok{c}\NormalTok{(}\StringTok{"Winter"}\NormalTok{, }\StringTok{"Spring"}\NormalTok{, }\StringTok{"Summer"}\NormalTok{, }\StringTok{"Fall"}\NormalTok{))}

\NormalTok{bike}\SpecialCharTok{$}\NormalTok{weekday }\OtherTok{\textless{}{-}} \FunctionTok{factor}\NormalTok{(}\FunctionTok{c}\NormalTok{(}\StringTok{"Sunday"}\NormalTok{, }\StringTok{"Monday"}\NormalTok{, }\StringTok{"Tuesday"}\NormalTok{, }\StringTok{"Wednesday"}\NormalTok{, }\StringTok{"Thursday"}\NormalTok{, }\StringTok{"Friday"}\NormalTok{, }\StringTok{"Saturday"}\NormalTok{)}
\NormalTok{                       [bike}\SpecialCharTok{$}\NormalTok{weekday.code }\SpecialCharTok{+} \DecValTok{1}\NormalTok{],}
                       \AttributeTok{levels =} \FunctionTok{c}\NormalTok{(}\StringTok{"Sunday"}\NormalTok{, }\StringTok{"Monday"}\NormalTok{, }\StringTok{"Tuesday"}\NormalTok{, }\StringTok{"Wednesday"}\NormalTok{, }\StringTok{"Thursday"}\NormalTok{, }\StringTok{"Friday"}\NormalTok{, }\StringTok{"Saturday"}\NormalTok{))}

\NormalTok{bike}\SpecialCharTok{$}\NormalTok{weather }\OtherTok{\textless{}{-}} \FunctionTok{factor}\NormalTok{(}\FunctionTok{c}\NormalTok{(}\StringTok{"Clear/Partly Cloudy"}\NormalTok{, }\StringTok{"Mist"}\NormalTok{, }\StringTok{"Rain or Snow"}\NormalTok{)[bike}\SpecialCharTok{$}\NormalTok{weathersit.code],}
                       \AttributeTok{levels =} \FunctionTok{c}\NormalTok{(}\StringTok{"Clear/Partly Cloudy"}\NormalTok{, }\StringTok{"Mist"}\NormalTok{, }\StringTok{"Rain or Snow"}\NormalTok{))}
\end{Highlighting}
\end{Shaded}

\begin{itemize}
\item
  Data should be treated as categorical are season.code (season),
  holiday.code, weekday.code (weekday) and weathersit.code.
\item
  List of variables should be treated as categorical variables:
\end{itemize}

\begin{enumerate}
\def\labelenumi{\arabic{enumi}.}
\tightlist
\item
  season.code (season): there are four seasons in a year and having it
  listed as a categorical variable makes sense.
\item
  holiday.code: this is binary variable because it can only be `yes' or
  `no'
\item
  weekday.code (weekday): when dealing with weekday, it is best to list
  weekday as categorical variable to determine day(s) with most bike
  usage. An argument can be made to used this as numerical variable
\item
  weathersit.code: each inputs in the variable has its own definition of
  a weather characteristic which cannot be expressed as a numeric value
\end{enumerate}

\hypertarget{task-3-12-points}{%
\subsection{Task 3 (12 points)}\label{task-3-12-points}}

Create a new variable called \texttt{workday} with values of
\texttt{Yes} if the day is indeed a workday and \texttt{No} if it is
either a weekend or a holiday. Describe one advantage and one
disadvantage in including \texttt{workday} in your model.

\begin{center}\rule{0.5\linewidth}{0.5pt}\end{center}

\begin{Shaded}
\begin{Highlighting}[]
\NormalTok{bike }\OtherTok{\textless{}{-}}\NormalTok{ bike }\SpecialCharTok{\%\textgreater{}\%}
  \FunctionTok{mutate}\NormalTok{(}\AttributeTok{workday =} \FunctionTok{case\_when}\NormalTok{(}
\NormalTok{    weekday.code }\SpecialCharTok{==} \DecValTok{0} \SpecialCharTok{|}\NormalTok{ holiday.code }\SpecialCharTok{==} \DecValTok{1} \SpecialCharTok{\textasciitilde{}} \StringTok{"No"}\NormalTok{,}
\NormalTok{    weekday.code }\SpecialCharTok{==} \DecValTok{6} \SpecialCharTok{|}\NormalTok{ holiday.code }\SpecialCharTok{==} \DecValTok{1} \SpecialCharTok{\textasciitilde{}} \StringTok{"No"}\NormalTok{, }
    \ConstantTok{TRUE} \SpecialCharTok{\textasciitilde{}} \StringTok{"Yes"}\NormalTok{)}
\NormalTok{    )}

\CommentTok{\# write.csv(bike, file = "/Users/brenthoang/Library/CloudStorage/OneDrive{-}BentleyUniversity/MA 380/Project 2/bike.csv", row.names = TRUE) {-}{-} check logic  }
\end{Highlighting}
\end{Shaded}

\begin{itemize}
\tightlist
\item
  advantage: may not need to include both weekday.code and holiday.code
  into final model to avoid model complexity
\item
  disadvantage: dismiss the true relationship between weekday and
  holiday to bike usage. maybe people use bike more during holiday than
  during the weekend.
\end{itemize}

\hypertarget{task-4-30-points}{%
\subsection{Task 4 (30 points)}\label{task-4-30-points}}

Conduct an exploratory data analysis on the information you have
available with a focus on answering some of the key questions that the
town's administration has. Select \textbf{three} graphs and, for each
one of them, explain what modeling decisions it supports.

\begin{center}\rule{0.5\linewidth}{0.5pt}\end{center}

\begin{Shaded}
\begin{Highlighting}[]
\NormalTok{bnum.hour }\SpecialCharTok{+}\NormalTok{ bnum.temp }\SpecialCharTok{+}\NormalTok{ bnum.weather}
\end{Highlighting}
\end{Shaded}

\begin{verbatim}
## `geom_smooth()` using method = 'gam' and formula = 'y ~ s(x, bs = "cs")'
\end{verbatim}

\begin{center}\includegraphics{ma380-pr02-hoa-qu_files/figure-latex/unnamed-chunk-6-1} \end{center}

\begin{itemize}
\tightlist
\item
  Interesting finding:
\end{itemize}

\begin{enumerate}
\def\labelenumi{\arabic{enumi}.}
\tightlist
\item
  Number of bike usage is less in holiday in general
\item
  There is little correlation between humidity and normalized
  temperature
\item
\end{enumerate}

\hypertarget{task-5-24-points}{%
\subsection{Task 5 (24 points)}\label{task-5-24-points}}

Explore the \textbf{mean-variance} relationship for the number of bikes
rented per hour. Provide a bivariate plot showing this relationship. For
each of the Poisson, Negative Binomial, and Gamma distributions use the
information in the mean-variance relationship to determine which of
these distributions would be most suitable for building a generalized
linear model.

\begin{center}\rule{0.5\linewidth}{0.5pt}\end{center}

\hypertarget{task-6-40-points}{%
\subsection{Task 6 (40 points)}\label{task-6-40-points}}

Based on your responses to the previous tasks select an initial model
(write it down here) and then search for a good model of the number of
bikes rented each hour. Select your final model and perform a thorough
diagnostic analysis.

\begin{center}\rule{0.5\linewidth}{0.5pt}\end{center}

\hypertarget{task-7-16-points}{%
\subsection{Task 7 (16 points)}\label{task-7-16-points}}

For a general audience interpret your final model from the previous
task.

\begin{center}\rule{0.5\linewidth}{0.5pt}\end{center}

\hypertarget{task-8-8-points}{%
\subsection{Task 8 (8 points)}\label{task-8-8-points}}

Some variables were not included in your final model. Select two of them
and explain why you did not include them. Back up your argument with
either a table or a graph.

\begin{center}\rule{0.5\linewidth}{0.5pt}\end{center}

\hypertarget{task-9-50-points}{%
\subsection{Task 9 (50 points)}\label{task-9-50-points}}

Write a short summary of your findings that you would share with the
town administrators. Be sure to address a general audience and to focus
your recommendations on solving the business problem they face.

Your written comments should not exceed 750 words. You may include two
graphs and/or tables to support your arguments.

\begin{center}\rule{0.5\linewidth}{0.5pt}\end{center}

\end{document}
